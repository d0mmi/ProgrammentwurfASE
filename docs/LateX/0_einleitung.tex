%% TODO Einleitung / Erklärung auch dass UI nur Optional ist um Funktion anzuschauen
%% TODO Rechtschreibung prüfen
\chapter{Einleitung}

In diesem Programmentwurf wurde ein User Verwaltungs System Backend für einen beliebigen Game Server entworfen.
Das bedeutet dieses System soll auf die Datenbank eines Game Servers mit Usern zugreifen und diese über eine API Schnittstelle verwalten.
Um das testen zu vereinfachen werden beim ersten starten der Anwendung ein initialer root Admin User und 3 Beispiel Ränge angelegt.
Die Login Daten des root Users stehen in der docker-compose Datei.
Es wurde auch ein simples UI entworfen um die Funktionalität der API Schnittstelle einfacher überprüfen zu können.
Dieses UI gehört aber nicht zum eigentlichen Programmentwurf und sollte daher als von der Bewertung ausgeschlossen betrachtet werden.
In diesem UI werden nur die Funktionen angezeigt für die der User auch genügend Rechte hat, daher sieht man als Nicht-Admin nicht alle Funktionen des UIs.

\section{Installation}

\subsection{Requirements}

Benötigte Software zum ausführen des Programmentwurfs:

\begin{itemize}
    \item Java 11
    \item Docker Desktop
    \item docker-compose
\end{itemize}

\subsection{Starten der Anwendung}
Befehle im root Order der Repo ausführen.
Das erste bauen könnte etwas länger dauern.

Bauen und ausführen: docker-compose up --build
\newline
Starten ohne neu bauen: docker-compose start
\newline
Stoppen: docker-compose stop
\newline
Stoppen + alles löschen: docker-compose down --volumes
\newline

Nach dem starten sollte gewartet werden bis in der Console der Datenbank steht: mysqld: ready for connections.
Dadurch kann sichergagangen werden, dass das sich Backend erfolgreich mit der Datenbank verbinden kann.

\subsection{Benutzen der Anwendung}

Default Admin User: email: root@example.com pw: example




Frontent: http://localhost:3000/
Wenn noch keine Reports / Bans in der DB sind zeigt das UI dauerhaft einen Ladebalken auf diesen Seiten an.
Wenn ein User einen neuen Rang bekommt, aber noch im UI angemeldet ist: Einmal neu im UI Anmelden damit eine neue Session mit dem aktuellen Rang ausgestellt wird.
\newline
API-Server: http://localhost:7001/
\newline
Datenbank: localhost:3306