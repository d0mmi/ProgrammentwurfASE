\chapter{Domain Driven Design}
\section{Analyse der Ubiquitous Language}

%% „allgegenwärtige Sprache“
%% Gleiche Begriffe in Domäne und Sourcecode
%% Jede Domäne besitzt eine eigene Fachsprache
%% Verstehen, warum diese Fachsprache gesprochen wird
%% Nicht versuchen, diese Fachsprache in „eigene“ Begriffe zu übersetzen
%% Ubiquitous Language bezeichnet die von Domänenexperten und Entwicklern gemeinsam im Projekt verwendete Sprache

%% Die Ubiquitous Language soll die Kluft reduzieren, indem Domänenexperten und Entwickler eine gemeinsame Sprache finden, die
%% - Alle relevanten Konzepte, Prozesse und Regeln der Domäne beschreibt
%% - Zusammenhänge verdeutlicht
%% - Mehrdeutigkeiten und Unklarheiten beseitigt

\begin{center}
    \begin{tabular}{ | l | l | }
        \hline
        Wort & Bedeutung                   \\ \hline
        Test & Test ist sehr toll          \\ \hline
        User & User spielen auf dem Server \\
        \hline
    \end{tabular}
\end{center}


\section{Analyse und Begründung der verwendeten Muster}

\subsection{Value Objects}

%% Value Objects (VO) sind einfache Objekte ohne eigene Identität
%% VO sind unveränderlich (immutable)
%% Ein VO kapselt ein „Wertkonzept“ und wird nur durch seine Eigenschaften oder Werte beschrieben -> daher: Value Object
%% Daraus folgt: zwei VO sind gleich, wenn sie die selben Werte haben

%% Vorteile von Unveränderlichkeit
%% Wenn ein Objekt gültig konstruiert wurde, kann es danach nicht ungültig werden
%% ->Einhalten von Invarianten der Domäne im Code wird sehr einfach
%% Unveränderliche Objekte sind frei von Seiteneffekten
%% -> Code ist weniger anfällig für ungewolltes Verhalten

%% Vorteile von Value Objects
%% Verbessern die Deutlichkeit und Verständlichkeit durch Modellierung von fachlichen Domänenkonzepten
%% Kapseln Verhalten und Regeln
%% Unveränderlich (frei von Seiteneffekten , beispielsweise Aliasing)
%% Selbst-Validierend
%% Leicht testbar

%% Implementierung von Value Objects in Java
%% lasse ist „final“ deklariert (Vererbung unterdrücken)
%% Alle Felder sind „blank final“ deklariert
%% equals() und hashCode() sind passend überschrieben
%% Ist nach Konstruktion in gültigem Zustand (andernfalls muss Konstruktion fehlschlagen)
%% Keine Setter oder andere Methoden, durch die Felder geändert werden können
%% Alle Methoden mit Rückgabewert liefern entweder:
%% Unveränderliche Rückgabewerte (immutable) oder
%% Defensive Kopien


\subsection{Entities}

%% Eine Entity unterscheidet sich in drei wesentlichen Punkten von einem VO:
%% Sie hat eine eindeutige ID innerhalb der Domäne
%% Zwei Entities sind verschieden, wenn sie verschiedene IDs haben; ihre Eigenschaften sind unerheblich
%% Eine Entity hat einen Lebenszyklus und verändert sich während ihrer Lebenszeit
%% Wie auch VO sollen Entities die Einhaltung der für sie geltenden Domänenregeln (Invarianten) forcieren:
%% Es darf nicht möglich sein, eine Entity mit ungültigen Werten zu erzeugen
%% Es darf nicht möglich sein, eine Entity nach Konstruktion in einen ungültigen Zustand zu versetzen
%% Entities sollten so viel Verhalten wie möglich in VO auslagern
%% (mindestens) die öffentlichen Methoden einer Entity sollten Verhalten beschreiben und nicht nur einfache getter/setter darstellen

%% Strategien für einzigartige Identitäten
%% Es gibt mehrere Strategien, um eine Entity eindeutig identifizierbar zu machen
%% Jede Strategie hat mehr oder weniger ausgeprägte Vorteile und Nachteile
%% Die jeweils passende Strategie hängt (wie immer) von den Anforderungen ab
%% Es spricht nichts dagegen, mehrere Strategien in einer Anwendung zu verwenden
%% Grundsätzliche Unterscheidung: 
%% natürliche Schlüssel und Surrogatschlüssel


%% Value Objects vs Entities
%% http://prntscr.com/w4mgu2



\subsection{Aggregates}

%% Wenn die Domäne maßstabsgetreu modelliert wird, findet man viele Entities und VO, die große Objektgraphen mit oft bidirektionalen Abhängig-keiten bilden
%% Das wird schnell ungemütlich:
%% Wahrscheinlichkeit nicht eingehaltener Regeln steigt
%% Verstärkt Kollisionen beim gleichzeitigen Bearbeiten
%% Performance-Einbußen durch Warten auf Sperrenfreigabe
%% Lange Wartezeiten beim Laden und Speichern
%% Aggregate gruppieren die Entities und VO zu gemeinsam verwalteten Einheiten
%% Jede Entity gehört zu einem Aggregat – selbst wenn das Aggregat nur aus dieser Entity besteht
%% Aggregate reduzieren die Komplexität der Beziehungen zwischen den Objekten
%% Das Aggregat wird immer als Einheit betrachtet und verwaltet (geladen und gespeichert)
%% Es gibt klare Regeln, wie außenstehende Objekte mit dem Aggregat interagieren dürfen
%% In jedem Aggregat übernimmt eine Entity die Rolle der Aggregate Root Entity bzw. des Aggregat Root (AR)
%% Alle Zugriffe auf das Aggregat müssen über das AR erfolgen
%% Auch Zugriffe auf die inneren Elemente des Aggregat
%% Langfristige direkte Referenzen auf innere Elemente sind nicht erlaubt
%% Nur temporäre Referenzen während einer Berechnung
%% Eines pro Aggregat
%% Das AR kann als eine Art 
%% „Türsteher“ alle Zugriffe auf 
%% das Aggregat kontrollieren
%% Zentrale Stelle zur Über-
%% wachung der Domänenregeln
%% Beispiel: Der Gesamtpreis einer Erstbestellung darf nicht mehr als 100 € betragen
%% aggregatsinterne Angelegenheiten bleiben „in der Familie“
%% Zugriff auf Aggregat nur über Aggregate Root
%% Wenn das AR Referenzen auf innere Objekte herausgeben muss, sollten das immer defensive Kopien oder Immutable-Dekorierer sein
%% Aussenwelt muss eine Methode auf dem AR aufrufen, wenn der innere Zustand des AR verändert werden soll
%% Das Aggregat sorgt dafür, dass sein Zustand immer den Domänenregeln entspricht
%% Alle Änderungen gehen über den AR und sind daher bekannt
%% Wenn die Außenwelt den AR „vergisst“, ist das gesamte Aggregat nicht mehr erreichbar

%% Zusammenfassung Aggregate
%% Aggregate sind Zusammenfassungen von Entities und Value Objects
%% Jedes Aggregat bildet eine eigene Einheit (auch für Create, Read, Update, Delete – CRUD)
%% Wird immer vollständig geladen und gespeichert
%% Aggregate
%% entkoppeln die Objektbeziehungen
%% bilden natürliche Transaktionsgrenzen
%% sichern Entity-/VO-übergreifende Domänenregeln zu
%% Aggregate sind mächtig, aber auch schwierig


\subsection{Repositories}

%% Kapseln die Logik für das Persistieren und Erzeugen von Entities, Value Objects und Aggregates
%% Halten das Modell frei von „accidental complexity“
%% Repositories vermitteln zwischen der Domäne und dem Datenmodell
%% Sie stellen der Domäne Methoden bereit, um Aggregates aus dem Persistenzspeicher zu lesen, zu speichern und zu löschen
%% Der konkrete technische Zugriff (accidental complecity) auf den Speicher (relationale DB, NoSQL, XML-Dateien usw.) wird vom Repository verborgen
%% Dadurch bleibt die Domäne von technischen Details unbeeinflusst
%% Repositories arbeiten direkt mit Aggregates zusammen
%% je Aggregate existiert also typischerweise ein Repository
%% Repositories liefern immer die Aggregate Roots (und damit den Zugriff auf den Rest) zurück
%% Die Definition der Repositories ist Teil des Domain Code
%% Die Implementierung findet „außerhalb“ statt
%% Die Methoden des Repository-Interface werden in der Sprache der Domäne benannt
%% Der Klassenname kann „Pure Fabrication“ sein
%% Der Rest der Anwendung muss eine bestimmte Aggregate Root anhand ihrer Eigenschaften finden können
%% Diese Abfragen sind die wichtigste Aufgabe des Repositories
%% Die Abfragen sollten genau zu den Aufgaben der Domäne passen
%% Selbst wenn es eine allgemeine Abfrage (über Kriterien) gibt, lohnen sich die speziellen Methoden
%% Das Repository kann für die Erstellung von Identifikationen (IDs) für neue Root Entities zuständig sein
%% Kann zusätzliche speicherseitige Prüffelder bei Veränderungen setzen („zuletztGeändertAm“)
%% Kann für Unit Tests leicht gemockt werden
%% Kann für Integrationstests ohne Datenbank durch eine In-Memory-Implementierung ersetzt werden
%% Die Implementierung erfolgt normalerweise in einer technischen Schicht der Anwendung (DB, Infrastruktur, …) und nicht innerhalb des Domänenmodells
%% Keine Vermischung von essential und accidental complexity

%% Zusammenfassung Repositories
%% Repositories bieten dem Domain Code Zugriff auf persistenten Speicher
%% In der Granularität von Aggregates
%% Direkter Zugriff immer nur auf die Aggregate Roots
%% Repositories verbergen die konkrete Speicher- technologie vollständig vor dem Domain Code
%% Anti-Corruption-Layer zur Persistenzschicht
%% Repositories bieten passend für die Domäne Abfragemöglichkeiten auf den Datenbestand
%% Adapter zwischen Anwendung und Datenbank


%% Factories 
Factories haben nur einen einzigen Zweck:
das Erzeugen von Objekten
Factories sind ein allgemein nützliches Konzept, unabhängig von DDD
Wenn die Logik für das Erzeugen einer Entity, eines Aggregates oder eines VO komplex wird, kann dies den eigentlichen Zweck des Objekts verschleiern (Verletzung des Single-Responsibility-Prinzips)
Factories helfen, indem sie dem Objekt die Verantwortung für seine Konstruktion abnehmen; dadurch kann sich das Objekt auf sein Verhalten konzentrieren
Der Begriff „Factory“ wird in OOP mehrdeutig verwendet. Es bezeichnet sowohl:
Das allgemeine Konzept einer Factory:
irgendein Objekt oder irgendeine Methode zur Erzeugung andere Objekte als Konstruktor-Ersatz
spezielle Erzeugungs-Muster
Factory Method
Abstract Factory
Im DDD meint man mit Factory normalerweise das „allgemeine Konzept“
Allgemein ist eine Factory irgendein Objekt/ irgendeine Methode als Konstruktor-Ersatz (siehe unten)




\subsection{Domain Services}

%% Ein Domain Service hat zwei Haupt-Einsatzzwecke:
%% Abbildung von komplexem Verhalten, Prozessen oder Regeln der Problemdomäne, die nicht in eindeutig einer bestimmten Entity oder einem bestimmten VO zugeordnet werden können
%% Definition eines „Erfüllungs-Vertrages“ für externe Dienste, damit das Domänenmodell nicht mit unnötiger „accidental complexity“ belastet wird
%% Manchmal kann ein bestimmtes Verhalten oder eine bestimmte Regel nicht eindeutig einer Entity oder einem VO zugeordnet werden
%% Beispiel:
%% Berechnung der Zahlungsmoral eines Kunden
%% Benötigt Kunde
%% Benötigt Rechnungen
%% Benötigt Kontenbewegungen
%% Die Domäne kann zur Erfüllung der Anforderungen auf externe Unterstützung angewiesen sein, beispielsweise durch einen Webservice, der von einer Fremdanwendung bereitgestellt wird
%% Innerhalb des Domänenmodells kann dazu ein Domain Service als Vertrag (Interface) definiert werden
%% Außerhalb des Domänemodells kann  dann ein „Dienstleister“ (beispielsweise in der Infrastruktur-Schicht – siehe Onion Architecture) diesen Vertrag implementieren und die benötigten Funktionen bereitstellen 
%% urch die Definition eines Vertrages kann das Domänenmodell vorgeben, was für ein Ergebnis erwartet wird („CreditWorthiness „) und welche Daten es zur Erfüllung des Vertrages bereitstellt („Customer“)
%% „fachlich“ ist dann alles relevante im Domänenmodell abgebildet – lediglich die technischen Details werden an eine Schicht außerhalb des Domänenmodells delegiert
%% Der Domain Service bezieht sich auf ein Domänenkonzept, das nicht natürlicherweise Teil einer Entity oder eines Value Object ist
%% Die Schnittstelle (die Methodensignaturen) verwendet die Begriffe des Domänenmodells
%% Ein- und Ausgabeparameter sind Entities und VO
%% Der Domain Service selbst ist zustandlos
%% Jede konkrete Instanz des Domain Service kann verwendet werden (alt oder neu)
%% Er darf aber (global sichtbare) Seiteneffekte haben









%% Zusammenfassung

%% Um die Entities und Value Objects technisch sauber implementieren zu können, benötigen wir unterstützende Strukturen
%% Domain Services enthalten Use Cases oder entkoppeln von Drittsystemen
%% Aggregates gruppieren Entities und vereinfachen die Beziehungen zwischen den Gruppen
%% Aggregate Roots stellen die primären Objekte dar
%% Repositories entkoppeln die Persistenz und machen die Aggregate Roots einfach auffindbar
%% Factories erzeugen Objektgraphen mit komplexen Konstruktionsregeln und entlasten den Konstruktor (Einhaltung Single Responsibility Principle)

