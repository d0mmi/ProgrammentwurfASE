\chapter{Programming Principles}
\section{Analyse und Begründung für SOLID}
%% 
%% Single Responsibility Principle
%%  - Eine Klasse sollte nur eine Ursache oder einen Grund haben sich zu ändern
%%  - Niedrige Komplexität und Kopplung
%%  - Jede Klasse sollte nur eine Zuständigkeit haben
%%  - Eine Klasse erhält eine klar definierte Aufgabe
%%  - Komplexeres Verhalten entsteht durch Kombination mehrerer Objekte
%%  - Eine Klasse enthält Achsen auf der sich Anforderungen ändern können
%%  - Jede Zuständigkeit fügt eine weitere Achse hinzu
%%  - Jede Klasse sollte nur eine Achse haben

\subsection{Single Responsibility Principle}
Das Single Responsibility Principle sagt aus, dass eine Klasse nur eine Zuständigkeit haben sollte.
Somit hat jede Klasse eine klar definierte Aufgabe und es entsteht eine niedrige Komplexität des Codes.
Und um so niedriger die Komplexität des Codes desto besser lässt er sich warten und erweitern.

%% Open Closed Principle
%%  - Elemente der Software wie Klassen, Module, Funktionen sollten:
%%      - offen für Erweiterungen
%%      - geschlossen für Änderungen 
%%  - Erweiterungen durch Vererbung bzw.Implementierung von Interfaces
%%  - Neue Unterklasse mit angepasstem Verhaltenergänzen
%%  - Bestehender Code wird nicht geändert
%%  - Abstraktionen fördern die Erweiterbarkeit
%%  - Software ist nie immun gegen Änderungen
%%  - Der Entwickler entscheidet 
%%      - welche Erweiterungen möglich sind
%%      - was durch Änderungen ergänzt werden soll
%%  - Stabilität einer Klasse ist ausschlaggebend

\subsection{Open Closed Principle}
Durch das Open Closed Principle wird die Software offen für Erweiterungen aber geschlossen für Änderungen.
Das bedeutet der Code wird nur durch Vererbung bzw. Implementierung von Interfaces erweitert.
Dies führt dazu das bestehender Code nicht geändert wird, wodurch die Anwendung für eine gute kompatibilität über die Versionen sorgt.
Hierbei ist es wichtig viele Abstraktionen zu nutzen um die Erweiterbarkeit zu fördern.

%% Liskov Substitution Principle
%%  - Objekte eines abgeleiteten Typs müssen als Ersatz für Instanzen ihres Basistyps funktionieren ohne die Korrektheit des Programms zu ändern
%%  - Starke Einschränkung der Ableitungsregeln
%%  - Führt zur Einhaltung von Invarianzen
%%  - Invarianzen von Klassen berücksichtigen:
%%      - Abgeleitete Typen müssen schwächere Vorbedingungen haben
%%      - Abgeleitete Typen müssen stärkere Nachbedingungen haben
%%  - Design by Contract kann helfen Verstöße zufinden
%%  - Ableitung in OOP ist mehr eine "verhält sich wie" Beziehung anstatt einer "ist ein" Beziehung

\subsection{Liskov Substitution Principle}
Im Liskov Substitution Principle werden Ableitungsregeln stark eingeschränkt was zu einer Einhaltung von Invarianzen führt.
Hierbei müssen abgeleitete Typen schwächere Vorbedingungen und stärkere Nachbedingungen haben.
Durch dieses Prinzip ergeben sich somit im OOP \glqq verhält sich wie\grqq{} Beziehungen statt den üblichen \glqq ist ein\grqq{} Beziehungen.
Somit kann man sich auf ein bestimmtes Verhalten der abgeleiteten Typen verlassen, wenn man das Verhalten des Basistyps kennt.

%% Interface Segregation Principle
%%  - Anwender sollten nicht von Funktionen abhängig sein, die sie nicht nutzen
%%  - Schwere(fat) Interfaces und Klassen bündelnviel Funktionalität:
%%      - Ein Anwender einer Methode eines Interfaces istautomatisch abhängig von Änderungen ananderen Methoden des Interfaces
%%      - Ein Anwender hat Zugriff auf Methoden, die nichtfür ihn bestimmt sind
%%  - Interfaces passend zu den Anwenderngestalten
%%  - Führt dazu, dass Typen meist mehrere Interfaces implementieren
%%  - Ein Typ bedient dadurch mehrere Anwender
%%  - Schwere Klassen können nach wie vorbestehen, aber Anwender ist nur von leichten Interfaces abhängig
\subsection{Interface Segregation Principle}
Das Interface Segregation Principle sorgt dafür, dass Anwender nicht von Funktionen abhängig sind die sie gar nicht nutzen.
Dies wird ermöglicht indem man statt einzelnen großen schweren Interfaces mit vielen Funktionen, viele kleine leichte Interfaces mit wenigen Funktionen implementiert.
Somit kann man seine Klasse passend für den Anwendungszweck anpassen ohne unnötige extrafunktionen mit zu schleppen.

%% Dependency Inversion Principle
%%  - Klassischerweise sind High-Level Module von Low-Level Modulen abhängig:
%%      - Änderung in einer Low-Level Implementierungführt zu Änderung in High-Level Modul
%%      - Änderung in High-Level Modul führt eventuell zu Änderung in anderen Low-Level Modulen
%%  ⇒ Umkehrung (Inversion) der Abhängigkeit
%%  - High-Level Module sollten nicht von Low-Level Modulen abhängig sein. Beide sollten von Abstraktionen abhängen.
%%  - Abstraktionen sollten nicht von Detailsabhängig sein. Details sollten von Abstraktionen abhängen.
%%  - Regeln werden durch High-Level Modulevorgegeben
%%  - Low-Level Module sind Implementierungender Regeln
%%  - High-Level Module können wiederverwendet werden:
%%      - High-Level Module bilden ein Framework
%%  - Immer nur von Abstraktionen abhängig sein bedeutet:
%%      - Variablen oder Member sollten eine abstrakte Klasse oder ein Interface als Typ haben
%%      - Klassen sollten nur abstrakte Klassen oder Interfaces ableiten bzw. implementieren
%%      - Nur abstrakte Methoden implementieren
%%  - Beim initialen Aufbau der Anwendung werden Instanzen konkreter Klassen erzeugt
\subsection{Dependency Inversion Principle}
Im Dependency Inversion Principle wird die klassische Struktur in der High-Level Module von Low-Level Modulen abhängig sind umgekehrt.
Denn Abstraktionen sollten nicht von Details abhängig sein, daher werden hierbei die Regeln durch High-Level Module vorgegeben und in Low-Level Modulen Implementiert.
Dadurch bilden die High-Level Module ein Framework und es werden nur noch abtrakte Abhängigkeiten genutzt.
Ein großer Vorteil der dieses Prinzip ist die hohe Fexibilität der Software, denn Low-Level Module können somit einfach ausgetauscht werden ohne die High-Level Module zu beinflussen.


\section{Analyse und Begründung für GRASP} %%  (insb. Kopplung/Kohäsion)
%% 
%% General Responsibility Assignment Software Patterns
%%  - Basis Prinzipien auf denen Entwurfsmuster aufbauen
%%  - Ziel ist die Low Representational Gap (LRG) möglichst klein zu halten
%%      - Die Lücke zwischen gedachten Domänenmodellund Softwareimplementierung (Designmodell)sollte klein sein
%%  - Zuweisung von Verantwortlichkeiten bzw. Zuständigkeiten
%%  - Zuständigkeiten haben 2 Typen:
%%      - Ausführend bedeutet: Objekte erstellen, Objekte kontrollieren, Aktionen ausführen
%%      - Wissen über: gekapselte Daten, Beziehungen zu zugehörigen Objekten, ableitbare bzw. berechenbare Informationen
GRASP steht für General Responsibility Assignment Software Patterns, dies sind Basis Prinzipien auf denen Entwurfsmuster aufbauen.
Das Ziel hierbei ist es, die Low Representational Gap (LRG) möglichst klein zu halten.
Also die Lücke zwischen gedachten Domänenmodell und Softwareimplementierung (Designmodell).
%% 
%% Low Coupling:
%%  - Lose bzw. geringe Kopplung
%%  - Kopplung bzw. Coupling beschreibt dieBeziehungen zwischen Objekten
%%  - Kopplung ist ein Maß für die Abhängigkeitzwischen Objekten
%%  - Positive Effekte durch geringe Kopplung:
%%      + Geringere Abhängigkeit zu Änderungen in anderen Teilen
%%      + Einfacher testbar
%%      + Verständlicher, da weniger Kontext notwendig ist
%%      + Einfacher wiederverwendbar
%%  - Formen der Kopplung im Code z.B. in Java:
%%      - X implementiert Interface Y
%%      - X ist abgeleitet von Klasse Y (auch indirekt)
%%      - X hat ein Attribut vom Typ Y
%%      - X hat eine Methode mit Referenz zu Klasse Y (Parameter, lokale Variable oder Rückgabewert)
%%      - X verwendet eine statische Methode von Klasse Y
%%      - X verwendet eine polymorphe Methode von Klasse oder Interface Y
%%  ⇒ Komponenten werden austauschbar, wenn die Kopplung lose ist
%%  - Kopplung an konkrete oder abstrakte Datentypen (Klassen und Interfaces)
%%  - Kopplung verschiedener Threads (Gemeinsame Sperren bzw. Locks)
%%  - Kopplung durch Resourcen (Gemeinsame Dateien, Speicher, CPU)
%%  ⇒ Kopplung zu stabilen Komponenten weniger problematisch
\subsection{Low Coupling}
Die Kopplung ist ein Maß für die Abhängigkeit zwischen Objekten, wobei versucht wird eine möglichst geringe Kopplung zu erreichen.
Dadurch hat man eine geringere Abhängigkeit zu Änderungen in anderen Teilen des Codes und die Software wird einfacher testbar und wiederverwendbar.
Zudem wird die Software verständlicher, da weniger Kontext benötigt wird um einzelne Ausschnitte des Codes zu verstehen.
Kopplung entsteht in Java zum Beispiel bei der Implementierung von Interfaces,
durch das halten eines Attributs vom Typ einer anderen Klasse oder durch das Besitzen einer Methode mit Referenz zu einer anderen Klasse.
Durch eine lose Kopplung werden Komponenten austauschbar.
Man kann an konkrete oder abstrakte Datentypen gekoppelt sein, aber auch durch verschiedene Threads mit gemeinsamen Sperren oder durch Resourcen wie gemeinsame Dateien.
Dabei ist eine Kopplung zu stabilen Komponenten weniger problematisch.

%% 
%% High Cohesion:
%%  - Hohe bzw. starke Kohäsion
%%  - Kohäsion ist ein Maß für den Zusammenhalteiner Klasse
%%      - Beschreibt die semantische Nähe der Elemente einer Klasse
%%  - Hohe Kohäsion und Lose Kopplung als Fundament für idealen Code
%%  + Einfacheres und verständlicheres Design 
%%  + Komponenten werden wiederverwendbarer 
%%  - Semantische Nähe der Attribute und Methoden bestimmen
%%      - Semantik nur schwer automatisiert testbar
%%      - Menschliche Einschätzung notwendig
%%  - Automatisch bestimmte technische Metriken
%%      - Anzahl Attribute und Methoden einer Klasse
%%      - Häufigkeit der Verwendung der Attribute in allen Methoden
%%      - Nicht immer treffend
\subsection{High Cohesion}
Kohäsion ist ein Maß für den Zusammenhalt einer Klasse, es beschreibt die semantische Nähe der Elemente einer Klasse.
Hohe Kohäsion und Lose Kopplung sind das Fundament für idealen Code, da sie zu einem einfacheren und verständlicheren Design führen.
Die Semantische Nähe der Attribute und Methoden ist nur schwer automatisiert testbar und es ist Menschliche Einschätzung notwendig.
Zu den Automatisch bestimmbaren technischen Metriken gehören Anzahl Attribute und Methoden einer Klasse und Häufigkeit der Verwendung der Attribute in allen Methoden aber diese sind nicht immer ideal.
%% Information Expert: 
%%  - Allgemeine Zuweisung einer Zuständigkeit zu einem Objekt
%%  - Einfachste Möglichkeit
%%      - Das Objekt, das die Informationen besitzt, erhält die Verantwortung dafür
%%  - Befragung von Domänen- und Designmodell
%%      - Wenn im Designmodell eine passende Klass eexistiert wird diese verwendet
%%      - Ansonsten wird im Domänenmodell einepassende Repräsentation gesucht und dafür eineKlasse im Designmodell erstellt
%%  - Objekte sind zuständig für Aufgaben über diesie Informationen besitzen
%%      - Informationen können auch auf Teilexpertenverteilt sein
%%      - Experte sammelt Informationen von Teilexpertenum Aufgabe zu erledigen
%%  + Kapselung von Informationen 
%%  + Leichtere Klassen, da Businesslogik zu denDaten verteilt wird 
%%  - Kann zu Problemen mit anderen Prinzipienführen
%%  - Separation of Concerns kann eine Lösung sein
\subsection{Information Expert}
Hierbei geht es um eine allgemeine Zuweisung einer Zuständigkeit zu einem Objekt.
Die einfachste Möglichkeit ist es dem Objekt, das die Informationen besitzt, die Verantwortung dafür zu überreichen.
Wenn im Designmodell eine passende Klasse existiert wird diese verwendet, ansonsten wird im Domänenmodell eine passende Repräsentation gesucht und dafür eine Klasse im Designmodell erstellt.
Objekte sind dann zuständig für Aufgaben über die sie Informationen besitzen, wodurch eine kapselung von Informationen und leichtere Klassen enstehen.
Aber es kann zu Problemen mit anderen Prinzipien führen.
%% Creator:
%%  - Das Erzeuger-Prinzip legt fest, wer für die Erzeugung von Objekten zuständig ist
%%  - Ein Objekt der Klasse B ist zuständig für die Erzeugung von Objekten der Klasse A, wenn
%%      - B eine Aggregation von A ist
%%      - B enthält Objekte von A
%%      - B erfasst Objekte von A
%%      - B nutzt Objekte von A mit starker Kopplung
%%      - B hat sämtliche Informationen zur Initialisierungvon A (B ist Experte zur Erstellung von A)
%%  - Allgemein gehalten kommt ein Objekt als Creator eines anderen in Frage, wenn es zujedem erstellten Objekt eine Beziehung hat
%%      - Eine Komposition in UML deutet auf einen Creatorhin
%%  + Ein geeigneter Creator verringert dieKopplung von Komponenten 
\subsection{Creator}
Im Creator-Prinzip wird festgelegt wer für die Erzeugung von Objekten zuständig ist.
Allgemein gesehen kommt ein Objekt als Creator eines anderen in Frage, wenn es zu jedem erstellten Objekt eine Beziehung hat.
Ein geeigneter Creator verringert die Kopplung von Komponenten, was zu den bereits genannten Vorteilen führt.
%% Indirection:
%%  - Indirektion bzw. Delegation (<T>)
%%  - Kann Systeme oder Teile von Systemenvoneinander entkoppeln
%%  - Indirektion bietet mehr Freiheitsgrade als Vererbung bzw. Polymorphismus (Benötigt aber auch mehr Aufwand bzw. Code)
%%  - Schnittstelle ist auf den Anwendungszweck angepasst
%%  - Mehr Flexibilität
%%  - Komposition verschiedener Objekteerzielt das gewünschte Ergebnis
\subsection{Indirection}
Durch Indirection werden Systeme oder Teile von Systemen voneinander entkoppelt.
Es bietet mehr Freiheitsgrade als Vererbung oder Polymorphismus, aber benötigt auch mehr Aufwand.
Dadurch ensteht das gewollte Low Coupling wodurch die Software viel flexibler wird.
%% Polymorphism:
%%  - Polymorphismus
%%  - Behandlung von Alternativen abhängig von einem konkreten Typ
%%  - Grundlegendes OO Prinzip zum Umgang mit Variation
%%      - Methoden erhalten je nach Typ eine andere Implementierung
%%  - Vermeidung von Fallunterscheidungen
%%      - Kein If-Else bzw. Switch
%%      - Konditionalstruktur wird im Typsystem codiert
%%  - Abstrakte Klasse oder Interface als Basistyp
%%      - Interfaces binden den Anwender nicht an eine Klassenhierarchie
%%  - Führt zur Verwendung des EntwurfsmustersStrategie
%%  - Polymorphe Methodenaufrufe werden erst zur Laufzeit gebunden
%%  +  Einfacher erweiterbar
%%  +  Bestehende Implementierung muss nicht verändert werden
%%  +  Extrahierung von Frameworks wird vereinfacht
\subsection{Polymorphism}
Polymorphismus ist ein grundlegendes OO Prinzip zum Umgang mit Variation.
Dabei erhalten Methoden je nach Typ eine andere Implementierung, wodurch auf Fallunterscheidungen verzichtet werden kann.
Die Konditionalstruktur wird sozusagen im Typsystem codiert und es werden Abstrakte Klassen oder Interfaces als Basistyp genutzt.
Zudem werden Polymorphe Methodenaufrufe erst zur Laufzeit gebunden.
Mithilfe von Polymorphismus wird die Software einfacher erweiterbar und bestehende die Implementierung muss nicht verändert werden.
%% Controller:
%%  - Verarbeitung von einkommenden Benutzereingaben
%%  - Koordination zwischen Benutzeroberfläche und Businesslogik
%%      - Einziger Ansprechpartner der Benutzeroberfläche
%%  - Hauptsächlich Delegation zu anderen Objekten
%%      - Controller enthält keine Businesslogik
%%  - Zustand der Anwendung kann in Controller gehalten werden
%%      - Aktion deaktivieren, während eine andere läuft
%%  - System Controller
%%      - 1 Controller für alle Aktionen
%%      - Nur bei kleinen Anwendungen praktikabel
%%  - Use Case Controller
%%      - 1 Controller pro Use Case
%%      - Viele kleine Controller
\subsection{Controller}
Der Controller verarbeitet einkommende Benutzereingaben und koordiniert zwischen Benutzeroberfläche und Businesslogik.
Seine Hauptaufgabe ist die Delegation zu anderen Objekten und er enthält keine Businesslogik.
Zu den verschiedene Arten von Controllern gehören der System Controller, wobei nur ein Controller für alle Aktionen genutzt wird, der nur für kleine Anwendungen praktikabel ist
und der Use Case Controller, welcher einen Controller pro Use Case nutzt.
%% Pure Fabrication:
%%  - Reine bzw. völlige Erfindung
%%  - Reine Verhaltens- oder Arbeits- Klasse
%%      -  Klasse besitzt keinen Bezug zur Problemdomäne
%%  - Trennung zwischen Technologie und Problemdomäne
%%      - Kapselung von Algorithmen
%%  + Einfach wiederverwendbar auch außerhalbder Domäne 
%%  + Begünstigt high cohesion durch Kapselung spezieller Funktionalität 
%%  ⇒ Sollte möglichst wenig vorkommen
\subsection{Pure Fabrication}
Bei der Pure Fabrication besitzt eine Klasse keinen Bezug zur Problemdomäne, wodurch eine Trennung zwischen Technologie und Problemdomäne und eine Kapselung von Algorithmen entsteht.
Damit werden diese Softwareteile auch außerhalb der Domäne einfach wiederverwendbar und durch die Kapselung spezieller Funktionalität wird High Cohesion begünstigt.
%% Protected Variations:
%%  - Sicherung vor Variation
%%  - Kapselung verschiedener Implementierungen hinter einer einheitlichen Schnittstelle (API)
%%  - Ursprünglich bekannt als Information Hiding
%%  - Der Einfluss von Variabilität einzelner Komponenten soll nicht das Gesamtsystem betreffen
%%  - Polymorphie und Delegation sind gute Schutzmöglichkeiten
%%      - Wechsel der Implementierung ist nicht relevant für das Gesamtsystem
%%  - Stylesheets im Webumfeld
%%      - Schützt vor konkretem Aussehen
%%  - Spezifikation von Schnittstellen
%%      - Schützt vor Implementierungsdetails
%%  - Betriebssysteme und Virtuelle Maschinen
%%      - Schützen vor konkreter Hardware
%%  - Begrenzt auch SQL
%%      - Schützt vor konkreter Datenbank
\subsection{Protected Variations}
Mit der Kapselung verschiedener Implementierungen hinter einer einheitlichen Schnittstelle wird die Software vor Variation gesichert.
Der Einfluss von Variabilität einzelner Komponenten soll dadurch nicht das Gesamtsystem betreffen.
Mit Polymorphie und Delegation lässt sich ein System gut schützen.
Weitere Schutzöglichkeiten gibt es durch Stylesheets im Webumfeld, Spezifikation von Schnittstellen oder Betriebssystemen und Virtuellen Maschinen.
\section{Analyse und Begründung für DRY}
%%  - Don’t Repeat Yourself!
%%  - Anwendbar auf alles mögliche
%%      - Datenbankschemata
%%      - Testpläne
%%      - Buildsystem
%%      - Dokumentation
Die Abkürzung DRY steht für Don’t Repeat Yourself und kann auf alles mögliche wie zum Beispeil Datenbankschemata, Testpläne, Buildsysteme oder Dokumentation angewendet werden.
%%  - "Every piece of knowledge must have a single,unambiguous, authoritative representationwithin a system"
%%  - Es darf nur eine Quelle der Wahrheit geben
%%  - Alle anderen Quellen werden davon abgeleitet
%%  - Vergleichbar zu den Normalformen bei RDBMS
%%  - Mechanische Duplikation ist erlaubt
Es darf dann nur noch eine Quelle der Wahrheit geben und alle anderen Quellen werden davon abgeleitet.
Das ist Vergleichbar zu den Normalformen bei RDBMS.
%%  - Auswirkungen der Modifikation eines Teils haben eine definierte Reichweite
%%      - Keine unbeteiligten Teile sind betroffen
%%      - Alle relevanten Teile ändern sich automatisch
Die Auswirkungen der Modifikation eines Teils haben eine definierte Reichweite und betreffen keine unbeteiligten Teile,
wobei sich alle relevanten Teile automatisch ändern.
%%  - Singleton ist keine Umsetzung des DRY Prinzips
%%      - Die Anzahl eines automatisch erzeugten Objekts ist irrelevant
Für die Entstehung von dupliziertem Code gibt es drei Hauptgründe:
\subsection{Imposed Duplication}
%%  - Imposed Duplication
%%      - Auferlegte Duplikation
%%      - Entwickler glaubt die Duplikation ist unumgänglich
Hierbei handelte es sich um eine auferlegte Duplikation und der Entwickler glaubt die Duplikation ist unumgänglich.
\subsection{Inadvertent Duplication}
%%  - Inadvertent Duplication
%%      - Versehentliche Duplikation
%%      - Entwickler bemerkt die Duplikation nicht
Dies ist eine versehentliche Duplikation da der Entwickler die Duplikation nicht bemerkt,
dies kann aber teilweise durch diverse Tools vermieden werden indem sie erkannte Duplikate markieren.
\subsection{Impatient Duplication}
%%  - Impatient Duplication
%%      - Ungeduldige Duplikation
%%      - Entwickler ist zu faul die Duplikation zu beseitigen
Diese ungeduldige Duplikation ensteht durch Entwickler die zu faul sind die Duplikation zu beseitigen.
