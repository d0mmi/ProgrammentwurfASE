\chapter{Clean Architecture}

%% Plugins (DB / GUI) -> Adapters (Presenters, Controllers, Gateways) -> Application Code (Use Cases) -> Domain Code (Entities) -> Abstraction Code (Generic Entities (mathematische Konzepte))
%% Abstraction Code: 
%%  Mathematische Konzepte (z.B. Matrizen)
%%  Algorithmen und Datenstrukturen (z.B. Zelluläre Automaten)
%%  Abstrahierte Muster (z.B. Quantitäten)
%%  !!!!!Häufig nicht notwendig!!!!
%% 
%% Domain Code:
%%  Entities 
%%  Implementiert organisationsweit gültige Geschäftslogik
%%  Sollte sich am seltensten ändern
%% 
%% Application Code:
%%  Use Cases
%%  Implementiert die anwendungsspezifische Geschäftslogik
%%  Steuert den Fluss der Daten und Aktionen von und zu den Entities
%% 
%% Adapters:
%%  Diese Schicht vermittelt Aufrufe und Daten an die inneren Schichten
%%  Formatkonvertierungen
%%  Oftmals nur einfache Datenstrukturen, die hin- und hergereicht werden
%%  Anti-Corruption Layer
%% 
%% Plugins:
%%  Diese Schicht greift grundsätzlich nur auf die Adapter zu
%%  Enthält Frameworks, Datentransportmittel und andere Werkzeuge (Datenbank, API)
%%  Wir versuchen, hier möglichst wenig Code zu schreiben
%%  Hauptsächlich Delegationscode, der an die Adapter weiterleitet
%% 
%% Grundregeln der Clean Architecture
%%  ●Der Anwendungs- und Domaincode ist frei von Abhängigkeiten●Sämtlicher Code kann eigenständig verändert werden
%%  ●Sämtlicher Code kann unabhängig von Infrastruktur kompiliert und ausgeführt werden
%%  ●Innere Schichten definieren Interfaces, äußere Schichten implementieren diese
%%  ●Die äußeren Schichten koppeln sich an die inneren Schichten (Richtung Zentrum)

%% Konkrete Umsetzung
%% ●Nicht alle Klassenin einem Projekt
%% ●Schichtenbildung überPackages ist in Ordnung
%% ●Aber: keine Überprüfungdurch den Compiler
%% ●Lieber mehrere Projekte(„Multi-Projekt“)
%% ●Compiler findet nur Klassen–im eigenen Projekt–in referenzierten Projekten
%% Maven parent Pom

%% Ziel der Clean Architecture
%% Das Ziel der Clean Architecture ist, Code nur von langlebigerem Code abhängig zu machen
%% Wenn sich Technologien ändern müssen, kann die Anwendung unverändert bleiben

\section{Schichtarchitektur planen und begründen}