\chapter{Refactoring}

\section{Long Method in APIServer}

\subsection{Identifizierung}

Der erste Code Smell den ich im Refactoring fand war der Lange Konstruktor der APIServer Klasse.
Da sollten vor dem Starten des Servers verschiedene Dinge initialisiert und konfiguriert werden.
Da häufte sich aber über die Zeit immer mehr an, was zu diesem Code Smell führte.

\subsection{Begründung des Refactorings}

In diesem Refactoring wurde zuerst das Konfigurieren der Kontroller in eine extra Klasse mit dem Namen APIServerConfig ausgelagert.
Diese Klasse wird dann als Parameter des Konstruktors übergeben was es in Zukunft ermöglicht den Server einfacher mit verschiedenen Konfigurationen zu starten.
Danach wurde der AccessManager in die Klasse APIServerAccessManager und die Konfiguration der Endpunkte in die Klasse APIServerEndpointGroup ausgelagert,
wodurch der Code viel verständlicher und überschaubarer wird.
Zuletzt wurde das erstellen der Serverinstanz, für eine bessere Übersichtlichkeit, noch in eine extra Funktion verschoben.
Dadurch wurde der Code insgesammt auf viele kleinere und logische Bausteine verteilt wodurch er einfacher zu verstehen ist.
Zudem konnten durch das aufteilen einfacher sinnvolle Namen für die Methoden der einzelnen Schritte gefunden werden.


Commit zum Refactoring: 04e489aab4cc16ecb07dde6ed08f51ef80c86231

\section{Duplicated Code}

\subsection{Identifizierung}

Ein weiterer Code Smell ist Duplicated Code, dieser wurde in der update Methode der diversen TableWraper Klassen in der Plugin-Database-Schicht gefunden.
Allgemein ist es der wichtigste Code Smell da er am häufigsten vorkommt und meistens durch Unwissenheit entsteht.
Dabei ist die gleiche Code-Struktur an mehr als einerStelle im Code vorhanden.
Das bedeutet die gleiche Anweisung kommt mehrfach in einer Methode oder Klasse vor oder der ähnliche Code verteilt sich auf mehreren Methoden oder Klassen.
In meinem Fall wurde in fast allen Klassen auf die selbe Art ein SQL String gebaut, dies könnte aber auch an einer anderen Stelle einmalig für alle implementiert werden.

\subsection{Begründung des Refactorings}

Dieser Code könnte in Zukunft Auseinanderdriften wodurch der Wartungsaufwand unnötigerweise erhöht wird.
Durch das Auslagern des gemeinsamen Codes können neue Strukturen geschaffen und die Wiederverwendbarkeit des Codes stark erhöht werden.
Die Funktion wurde deshalb verallgemeinert und in die abstrakte Superklasse TableWraper ausgelagert.


Commit zum Refactoring: 07c1d0806ac9774b6f8914010928ec39a0de33f1


%% TODO Frage sollen alle Smells erklärt werden oder nur die "gefundenen?"