\chapter{Refactoring}

\section{Long Method in APIServer}

\subsection{Identifizierung}

Der erste Code Smell den ich im Refactoring fand war der Lange Konstruktor der APIServer Klasse.
Da sollten vor dem Starten des Servers verschiedene Dinge initialisiert und konfiguriert werden.
Da häufte sich aber über die Zeit immer mehr an, was zu diesem Code Smell führte.

\subsection{Begründung des Refactorings}

In diesem Refactoring wurde zuerst das Konfigurieren der Kontroller in eine extra Klasse mit dem Namen APIServerConfig ausgelagert.
Diese Klasse wird dann als Parameter des Konstruktors übergeben was es in Zukunft ermöglicht den Server einfacher mit verschiedenen Konfigurationen zu starten.
Danach wurde der AccessManager in die Klasse APIServerAccessManager und die Konfiguration der Endpunkte in die Klasse APIServerEndpointGroup ausgelagert,
wodurch der Code viel verständlicher und überschaubarer wird.
Zuletzt wurde das erstellen der Serverinstanz, für eine bessere Übersichtlichkeit, noch in eine extra Funktion verschoben.
Dadurch wurde der Code insgesammt auf viele kleinere und logische Bausteine verteilt wodurch er einfacher zu verstehen ist.
Zudem konnten durch das aufteilen einfacher sinnvolle Namen für die Methoden der einzelnen Schritte gefunden werden.

Commit: 04e489aab4cc16ecb07dde6ed08f51ef80c86231

%% TODO 1 Refactoring
